\documentclass[a4paper,twoside,12pt]{report}
% Richard Klein (2020,2021)

% Include Packages
%\usepackage[a4paper,inner=3.5cm,outer=2.5cm,top=2.5cm,bottom=2.5cm]{geometry}  % Set page margins
\usepackage{fullpage}
\usepackage{float}                  % Allows 'Here and Only Here' [H] for Floats
\usepackage{url}                    % \url{} command
\usepackage{charter}                  % Set font to Times
\usepackage{graphicx}               % \includegraphics
\usepackage{subfigure}              % Allow subfigures
\usepackage{amsmath}
\usepackage{amssymb}
\usepackage{amsthm}
\usepackage{booktabs}
\usepackage{parskip}
\usepackage[all]{nowidow}
\setnoclub[2]
\setnowidow[2]

% Referencing
% Provides \Vref and \vref to indicate where a reference is.
\usepackage{varioref} 
% Hyperlinks references
\usepackage[bookmarks=true,bookmarksopen=true]{hyperref} 
% Provides \Cref, \cref, \Vref, \vref to include the type of reference: fig/eqn/tbl
\usepackage{cleveref} 
% Setup Hyperref
\hypersetup{
  colorlinks   = true,              %Colours links instead of ugly boxes
  urlcolor     = blue,              %Colour for external hyperlinks
  linkcolor    = blue,              %Colour of internal links
  citecolor    = blue                %Colour of citations
}
% Names for Clever Ref
\crefname{table}{table}{tables}
\Crefname{table}{Table}{Tables}
\crefname{figure}{figure}{figures}
\Crefname{figure}{Figure}{Figures}
\crefname{equation}{equation}{equations}
\Crefname{equation}{Equation}{Equations}

% Wits Citation Style
\usepackage{natbib} \input{natbib-add}
\bibliographystyle{named-wits}
\bibpunct{[}{]}{;}{a}{}{}  % to get correct punctuation for bibliography
\setlength{\skip\footins}{1.5cm}
\newcommand{\citets}[1]{\citeauthor{#1}'s \citeyearpar{#1}}
\renewcommand\bibname{References}  

\pagestyle{headings}

\pagestyle{plain}
\pagenumbering{roman}

\renewenvironment{abstract}{\ \vfill\begin{center}\textbf{Abstract}\end{center}\addcontentsline{toc}{section}{Abstract}}{\vfill\vfill\newpage}
\newenvironment{declaration}{\ \vfill\begin{center}\textbf{Declaration}\end{center}\addcontentsline{toc}{section}{Declaration}}{\vfill\vfill\newpage}
\newenvironment{acknowledgements}{\ \vfill\begin{center}\textbf{Acknowledgements}\end{center}\addcontentsline{toc}{section}{Acknowledgements}}{\vfill\vfill\newpage}

\begin{document}
\onecolumn
\thispagestyle{empty}

\setcounter{page}{0}
\addcontentsline{toc}{chapter}{Preface}
\ 
\begin{center}
  \vfill
  {
  \huge \bf \textsc{The Development of Cyber Threat Intelligence System Framework for the Mining Industry}\\
  \large Subtitle\\[20pt]
  \large School of Computer Science \& Applied Mathematics\\
  \large University of the Witwatersrand\\[20pt]
  \normalsize
  MUHAMMAD UMER FAROOQ\\
  2925331\\[20pt]
  Supervised by Dr Helen Robertson\\[10pt]
  \today
  }

  \vfill
  \vfill
  \includegraphics[width=1.5cm]{images/wits}
  \vspace{10pt}\\
  % \small{Ethics Clearance Number: XX/XX/XX}\\[10pt]
  \small{A proposal submitted to the Faculty of Science, University of the Witwatersrand, Johannesburg,
in partial fulfilment of the requirements for the degree of Bachelor of Science with Honours}\\
\end{center}
\vfill
\newpage

\pagestyle{plain}
\setcounter{page}{1}

\phantomsection
\begin{abstract}
Abstract things....
\end{abstract}

\phantomsection
\begin{declaration}
I, Muhammad Umer Farooq, hereby declare the contents of this research proposal to be my own work.
This proposal is submitted for the Master of Science by Dissertation in Computer Science at the University of the Witwatersrand.
This work has not been submitted to any other university, or for any other degree.
\end{declaration}

\phantomsection
\begin{acknowledgements}
I would like to express my deepest gratitude to my supervisors, Dr. Helen Robertson from the School of Computer Science and Mathematics, and Dr. M. Ahsan Mahboob, the Head of Sibanye Stillwater Digital Mining Laboratory (DigiMine), for their invaluable guidance and support throughout my research journey.

Dr. Robertson’s expertise and insightful feedback were instrumental in refining my work, and her encouragement was a constant source of motivation. I am equally grateful to Dr. Mahboob for his mentorship and for providing me with the opportunity to collaborate within the DigiMine lab, where his leadership and profound knowledge in the field greatly enhanced my research experience.

Their unwavering support, constructive feedback, and encouragement have been crucial to the successful completion of this proposal. I am deeply appreciative of their time and efforts, and I feel privileged to have worked under their guidance.
\end{acknowledgements}


\phantomsection
\addcontentsline{toc}{section}{Table of Contents}
\tableofcontents
\newpage
\phantomsection
\addcontentsline{toc}{section}{List of Figures}
\listoffigures
\newpage
\phantomsection
\addcontentsline{toc}{section}{List of Tables}
\listoftables
\newpage
\pagenumbering{arabic}
% Chapter 1
\chapter{Introduction}
The Mining Industry, a vital part of the global economy, increasingly depends on digital technologies to improve operations, safety, and sustainability. However, this reliance on technology exposes the industry to significant cyber threats that can disrupt operations, compromise safety, and cause financial loss and environmental harm. Given these risks, the need for effective cyber threat detection in mining is more urgent than ever. Traditional and reactive cybersecurity measures often fail to address the complex and evolving threats specific to the mining sector. This has led to the growing importance of Cyber Threat Intelligence (CTI) systems, which proactively detect, analyze, and respond to potential threats. These systems collect and analyze data from multiple sources, helping organizations identify and mitigate risks before they cause damage. Despite potential threats, the urgency for effective cyber threat detection in mining has never been greater. Traditional reactive cybersecurity measures often fall short of addressing the sophisticated and evolving threats unique to the mining sector. This research aims to fill these gaps by designing a comprehensive Cyber Threat Intelligence System for the mining industry. The system will use advanced data analytics and machine learning to detect patterns and anomalies in threat data, improving the industry’s ability to defend against cyber-attacks. Additionally, this study will examine the ethical implications of implementing such a system, ensuring it meets both technical and ethical standards. By designing a CTI system tailored to the mining sector, this research will address a critical need and contribute to the broader field of cybersecurity.
\section{Problem Statement}
Problem Statement things...\\
\section{Research Questions}
Research Questions...\\
\section{Research Aims and Objectives}
Research Aims and Objectives...\\
\subsection{Research Aims}
Research Aims\\
\subsection{Objectives}
Research Objectives...\\
\section{Limitations}
\section{Overview}


% \subsubsection{This is a subsubsection}
% This is just a paragraph
% \subsection{A Subsection about Citation Style}
% Citations are important. Citation style for Computer Science is:
% \begin{itemize}
% \item When used in the text, use the authors with the date in brackets:\\ \citet{klein17} say very important things.
% \item When used as a reference after a face, put everything in brackets:\\ Import things are true \citep{klein17}.
% \end{itemize}

% \subsection{Compiling}
% Remember to compile multiple times to resolve references. Usually:
% \begin{verbatim}
% pdflatex file.tex
% bibtex file
% pdflatex file.tex
% pdflatex file.tex
% \end{verbatim}

% Chapter 2
\chapter{Background and Literature Review}
\section{Introduction}
\subsection{Background}
\subsection{Related Work}
% Chapter 3
\chapter{Research Methodology}
\section{Research design}
\section{Methods}
\subsection{Pre-Modelling Phase}
\subsection{Modelling Phase}
\subsection{Post-Modelling Phase}
\subsection{Experimental Setup}
\subsection{Optimization and Training Models}
\section{Limitations}
\section{Ethical Considerations}


% Chapter 4
\chapter{Schedule of Work}
% Chapter 5
\chapter{Conclusion}



\LaTeX\ decides how to place images. It also does the referencing for you as seen in \Cref{fig:thing1}. If you have subimages, they should have their own captions and labels -- look into the subfig or subfigure packages.

\begin{figure}[ht]
	\centering
	\includegraphics[width=0.1\linewidth]{images/wits}
	\caption{This is an image}
	\label{fig:thing1}
\end{figure}

Figure captions are at the bottom. Table title are at the top of the table as seen in \Vref{tab:tab1}. There is a package called BookTabs which is \textit{way} better for tables and you should learn how to use that instead.

\begin{table}[p]
	\centering
	\caption{Table Name}
	\label{tab:tab1}
\begin{tabular}{cc}
	\hline
	Col1 & Col2\\
	\hline\hline 
	R0,C0 & R0,C1 \\ 
	R1,C0 & R1,C1 \\ 
	\hline
\end{tabular} 
\end{table}

Usually let \LaTeX\ handle the placement of floats unless you \textit{really} need to force it to do something else. The \texttt{float} package used above allows you to use \texttt{H} as the placement which means \textit{here and only here}. When using the float package, the placement options are:
\begin{enumerate}
\item h -- a gentle nudge to place it here if possible
\item t -- top of a page
\item b -- bottom of a page
\item H -- here and only here, do not move it at all
\item p -- on its own page
\end{enumerate}


\chapter{Some Referencing Tricks}
CleverRef and VarioRef are helpful:
\begin{itemize}
	\item Normal Ref: See Figure \ref{fig:thing1}
	\item CleverRef: See \Cref{fig:thing1} and \Cref{tab:tab1}
	\item CleverRef+VarioRef: See \Vref{fig:thing1} and \Vref{tab:tab1}
\end{itemize}

% \chapter{IDE/Editors}
% Overleaf has a great online editor for latex. Use it. 

\appendix
\chapter{Extra Stuff}\label{app:extra}
\section{What is an appendix?}\label{app:whatis}

An appendix is useful when there is information that you need to include, but breaks the flow of your document, e.g. a large number of figures/tables may need to be shown, but maybe only one needs to be in the text and the rest are just included for completeness.

\nocite{*}

\bibliography{references}\addcontentsline{toc}{chapter}{References}
\end{document}
